Design verification was primarily conducted using the requirements and validation tables generated during the design phase.
In this section, we will cover the verification steps taken to validate our boards through manufacturing and testing.
An in-depth check list of all requirements is included in Appendix A.

\subsection{Control Board}
Once our PCB and components were received, we sped into manufacturing the Control board.
We did not have a stencil for our PCB, so we hand-soldered all our components.
Validation of our soldering included measuring isolation between the various voltage nets on the PCB and GND.
Once satisfied, we supplied 12V to the board using a lab power supply, and inspected it with an IR camera to find hotpots (indicative of short circuits).

\begin{center}
    \includegraphics[width=0.65\textwidth]{images/thermal_camera_testing.jpg}\\
    \it Image 3.1.1: Thermal camera image captured during testing\\
\end{center}

\subsection{Power Stage Board}
To test the power stage, we used the working control board to communicate with the DC/DC controller IC and E-meter IC.
We had noise issues on the I2C line when the DC/DCs were enabled.
To solve this issue, we swapped the 1 kOhm pull ups to 100 Ohm pull ups.
Our DC output was a sinusoidal waveform.
This was likely caused by our controller gain being set too low due to a misinterpretation in the application note for the DC/DC controller.

\begin{center}
    \includegraphics[width=0.8\textwidth]{images/testing_oscilloscope_output.png}\\
    \it Figure 3.2.1: DC output of the DC/DC controller\\
\end{center}
