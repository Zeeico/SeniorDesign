Modern EVs contain a host of sensors, actuators, temperature control devices etc. that all require specific input voltages.
Typical vehicles will have a single low-voltage battery that goes through two stages of conversion: once to distribute through the vehicle, and again to step down to the actual device requirement.
Our project aims to simplify this system by moving all control and conversion local to the components that use them.
The theory behind this is the desire of automotive companies to move to a “one power one communication” arrangement.

\begin{center}
    \includegraphics[width=0.8\textwidth]{./images/new_status_diagram.png}\\
    \textit{Figure 1.1: Status quo for power distribution/wire harness on moderns EVs}\\
\end{center}
\vspace{\baselineskip}

Applying our project onto such a vehicle produces three main improvements:
\begin{enumerate}
    \item Simplified harnessing leads to easier manufacturing and fewer transmission losses
    \item Switched converters for voltage step-downs lead to efficiency gains over linear regulators
    \item Improved data feedback allows better duty cycle analysis and system optimization
\end{enumerate}

\begin{center}
    \includegraphics[width=0.8\textwidth]{./images/new_status_diagram.png}\\
    \textit{Figure 1.2: Theoretical application of our project into the same vehicle as the previous image}\\
\end{center}
\vspace{\baselineskip}

Our long-term hope was to create a software-configurable device that could be applied in a broad area for many automotive families. 
In doing so, we would be able to create a cost effective solution by taking advantage of the inherent economies of scale already prevalent in the industry.

We made no changes to the high level structure of the project through the semester.

\subsection{High Level Requirements}
The success of this project is measured against the following criteria:
\begin{enumerate}
    \item Communication with the solution over Controller Area Network (CAN)
    \begin{enumerate}
        \item Independent control over each voltage output
        \item Voltage/current usage data received at 20Hz
        \item Visualization of voltage/current usage data by the user within the accuracy provided by the E-meter chip
    \end{enumerate}
    \item Supply up to 2A of current on all active rails simultaneously. Outputs should have less than $\pm$5\% voltage ripple and $\pm$5\% current ripple on the load compared to software setpoint
    \item Stay below 60° Celsius while providing 2A on all active rails for 60 minutes
\end{enumerate}
